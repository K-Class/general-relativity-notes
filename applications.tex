Now we get to the most interesting part of calculus: its applications. 
Here, we are going to focus on only a few applications, but there are many applications of calculus, across many fields, such as economics, sociology, physics, mathematics itself, chemistry, astronomy, and others. 
Calculus is key in many fields.

The main application we are going to look at is work, as defined by physics. 
Interestingly, there is an integral definition of work - that is, work can be defined using an integral.

\begin{equation*}
    \int^b_a f(x) \, dx
\end{equation*}

What does this mean? 
Well, first, $f(x)$ is a function representing force, where $x$ is displacement, which, as Newton's second law states, is $F = ma$ - that is, force is mass times acceleration. 
Second, $b - a$ should equal the displacement, or how much the object is moved. This is because $b$ and $a$ are what we are substituting for $x$, which is displacement.
This is easier to illustrate with a few problems.

For example, let's say we have a bowling ball, rolled with a force equal to the function $f(x) = 2x+3$ (where the resulting number is in Newtons). 
We want to find out how much work it takes to roll it from the start of our lane down to the end of the lane, 10 meters away.

Well, here, this is easy. All we do is plug in the numbers into our formula for work! 

\begin{equation*}
    \int^{10}_0 2x+3 \, dx
\end{equation*}

Remember that $a$ and $b$ together represent displacement. 
We're rolling the bowling ball from the start, $0$, to the end of the lane, $10$ meters away. 
So that represents $a$ and $b$ respectively. 
The force, $f(x)$, is simple to plug in. 
So now, we follow the rules for integrating. 
In this case, we get $2\frac{x^2}{2}+3x\mid^{10}_0$. 
Note that this last part, the line with the superscripts, is just a bit of notation, saying that we plug in $10$ and $0$ and subtract according to the rules of definite integrals. 
It just allows us to integrate and write that down. 

So now that we have this, we plug it in as we would for normal definite integrals, and so we get $(2\cdot\frac{10^2}{2}+3\cdot 10) - (2\cdot\frac{0^2}{2}+3\cdot 0)$ and then we simplify. 
The whole second half comes to zero, so we now have $2\cdot\frac{10^2}{2}+3\cdot 10$. 
Then, we have $2\cdot \frac{100}{2}+ 30$ or $2\cdot 50 + 30$ or $130$. 
Now we look at units. 
We have meters and newtons, so now when we multiply those two, we have joules (a unit of work). 
So our answer is $130$ joules. 
Note that it is very important to keep track of your units. If you don't know a unit of force, work, or distance, look it up! Think logically.

A quick note on Hooke's Law: when considering the work required to stretch a spring, you can assume that the force ($f(x)$) needed to stretch a spring a distance $x$ beyond its natural length is proportional to $x$. 
In other words, we have $f(x) = cx$ where $x$ is the distance the spring is stretched and $c$ is some constant.