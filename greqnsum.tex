And so, after quite a bit of text, we come to the meat of the matter: the Einstein field equations. Without further ado, here they are:
\begin{equation*}
R_{\mu\nu}-\frac{1}{2}g_{\mu\nu}R+g_{\mu\nu}\Lambda=\frac{8\pi G}{c^4}T_{\mu\nu}
\end{equation*}
While this may look deceptively simple, they are actually quite complicated. Here, $\mu$ and $\nu$ represent the dimensions of spacetime.
You may wonder how this is equations, plural. It may only look like one equation, but in fact, $\mu$ and $\nu$ are able to take 
four different values, representing the four different dimensions of spacetime. They show up at four different spots, resulting in 
sixteen variations of the same equation. However, six turn out to be duplicates, so there are ten Einstein field equations.

What are all the various letters and variables here? Well, we've already established what $\mu$ and $\nu$ are. $G$ we've already seen;
it is Newton's graviational constant. The speed of light is $c$, which is something you may already be familiar with, along with $\pi$
and the other constants on the right hand side. $R_{\mu\nu}$ is called the Ricci curvature tensor. $g_{\mu\nu}$ is the metric tensor.
(We will get into later what these different things mean.) $R$ is the curvature scalar. $T_{\mu\nu}$ is the stress energy momentum tensor.
Finally, $\Lambda$ is the cosmological constant.

This equation balances two things. On the left, the various terms represent the curvature of spacetime. On the right, the various terms
represent mass and energy. In other words, the equations basically say this: mass tells spacetime to curve, and spacetime tells mass
how to move.
